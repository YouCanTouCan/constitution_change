%%% Constitution for the Open SOurce Club at the Ohio State University
%%% distributed under a creative commons 0 (CC-0)
\documentclass{article}
\title{The Opensource Club Constitution}
\author{Isaac Jones -- Martin Jansche -- Michael Benedict}
\date{Version 1.0: 20 March, 2000\\Version 1.1: 24 March 2016\\Version 2.0: 11 Oct 2018}
\setcounter{secnumdepth}{0}
\usepackage[normalem]{ulem}
\usepackage{graphicx}
\begin{document}
	\maketitle

	\section{Preamble}

	It is important to recognize the danger that bureaucracy can get in the way of doing cool things.  So long as the requirements of the Ohio State University and the realistic needs of The Open Source Club are met, efforts should be made to minimize bureaucracy.

	As The Open Source Club has evolved it has become necessary to elect additional Officers in order to aid the further growth of The Open Source Club. While this is not an attempt to create a bureaucracy, it is necessary in order to delegate duties among Members in order to help The Open Source Club run smoothly. At the time of this writing there are currently three elected positions within The Open Source Club: President (or Benevolent Dictator), Vice President, and Treasurer. These three positions will continue to exist as long as they are required by the Ohio State University.

	\section{Article I - Definition}

	\subsection{Section 1 - Name}

	This is The Open Source Club at The Ohio State University. And in other places, known as The Open Source Club.

	\subsection{Section 2 - Purpose}

	The Open Source Club's purpose is to support and advocate open source software, and to create a community of excellent programmers, admins, and users. We do so through creating documentation, providing support, engaging within the community, and fighting cluelessness.

  \subsection{Section 3 - Public Relations and Endorsements}

  The Open Source Club currently maintains a public image in various forms: through social media, on our official website, at public events, and while conducting official meetings. While representing The Open Source Club, any Officer, Advisor(s) or Member shall not take any official or publicized stance on news, current events, or current affairs which infers or explicitly states the endorsement of The Open Source Club. Individuals may feel free to express their own opinions with the caveat that the views expressed are solely their own, and not endorsed by The Open Source Club.

  The Open Source Club may only give ``official'' endorsements to open source projects, events, or other appropriate intangibles which one or more Members is involved with. Any definition of open source, whether that be software, hardware, firmware, or something else qualifies.

	\subsection{Section 4 - Non Discrimination Policy}

	In recognition of the importance of welcoming diversity for the sake of creativity, and for the benefit of humanity, The Open Source Club welcomes all people. This means that The Open Source Club will not discriminate against any individual(s) for reasons of age, color, disability, gender identity or expression, national origin, race, religion, sex, sexual orientation, or veteran status.

	\section{Article II - Membership}

	\subsection{Section 1 - Categories}

	The Ohio State University guidelines demand that voting membership be limited to currently enrolled students. In order to meet this requirement, membership will be divided into voting and non-voting categories.

	\begin{itemize}
		\item Non-voting membership is obtained simply by attending one or more previous meetings and maintaining a subscription to The Open Source Club's mailing list.
		\item Voting membership requires that the individual meet all of The Ohio State University's requirements for voting membership, and The Open Source Club's non-voting membership requirements.
	\end{itemize}

	\subsection{Section 2 - Membership Privileges}

  Membership in The Open Source Club grants the following privileges:

  \begin{itemize}
    \item Access to The Open Source Club's computer systems and services
    \item Voting privileges to voting Members
    \item Volunteer opportunities on the behalf of The Open Source Club at various events
    \item Official association with The Open Source Club at The Ohio State University
  \end{itemize}

	\subsection{Section 3 - Voting}

  Any voting Member who is in good standing and is present (defined as ``being in synchronous communication and interacting with'' participants) at an ``official'' vote is eligable to cast a vote. The Open Source Club differentiates between the following standards of affirmation:

  \begin{itemize}
    \item Supermajories of 60\% (a supermajority of the vote)
    \item Pluralities (the plurality of the vote)
    \item Straight Majorities (the majority of the vote)
    \item Unanimity (a unanimous vote)
  \end{itemize}

  All of the preceeding qualify as acceptable standards of affirmation which The Open Source Club allows ``official'' votes to be taken on as is appropriate to the situation and context. However, all voting requirements in this Document supercede any other if applicable, regardless of context.

  Ties will be broken by a majority vote of the current Officers. If the current Officers' vote results in a tie, the Advisor(s) will act as the final arbiter between the Officers, and their decision will be incontestable.

	\subsection{Section 4 - Membership Removal and Reinstatement}

	Should cause for membership removal be raised, a Member can be removed by:

	\begin{enumerate}
		\item A unanimous vote of Officers.
		\item A supermajority vote of voting Members, providing there is at least a week's notice of the vote.
	\end{enumerate}

  Any reinstatement of a Member requires the same voting threshold to be met. However, once that Member is reinstated, they are not considered, and may never regain the status of ``in good standing''. All Members who have never been voted to be removed from membership are considered to be ``in good standing''.

	\section{Article III - Officers}

	\subsection{Section 1 - Officer Selection}

	Once a year, elections will take place for the Officer positions.  This election must be announced at least one week prior to the vote.  The individual who receives a voting plurality will attain that position for the next academic year, unless removed.

	\subsection{Section 2 - Officer Positions, Duties, Powers and Limitations.}

	There are three Officer positions required by the Ohio State University: The President (or Benevolent Dictator), the Vice President, and the Treasurer.  These three positions will continue to exist as long as they are required by the Ohio State University. Other Officer positions may be added and removed as needed, and may be open to election or appointed by the president. An Officer is required to be:

    \begin{itemize}
      \item A voting Member in good standing
      \item Meeting all of the requirements of an Officer per The Ohio State University's guidelines
      \item Subject to being listed on any given ``official'' document of The Open Source Club as their respective position and status
      \item Responsible for the public relations of The Open Source Club
      \item Responsible for the hardware and all assets of The Open Source Club
      \item The only people who can authorize funds to be spent
    \end{itemize}

	No Officer will necessarily have any control over software projects, since by the definition of open source software, code forks can happen when they need to, and no one therefore has ultimate power over a software project.

	\subsection{Section 3 - Officer Removal}

	Leadership is needed in any club and so Officers are required to be present at as many meetings as possible, and must make every effort to do so.  An Officer may be subject to removal if they neglect to attend a supermajority of meetings in a semester, or consecutive meetings in an amount which is equivalent to half of the scheduled meetings in a semester.

  However, it is also realized that extenuating circumstances are possible.  In order to eliminate doubts surrounding an Officers excuse for missing a meeting, any Officer who plans to skip a meeting must notify at least one of the other Officers that they will not be in attendance at least 24 hours in advance.

  If an Officer is in violation of the aforementioned attendance rules, then voting Members must take a supermajority vote as to whether or not said Officer will be allowed to keep their position. For all other cases of misconduct, the Officer will face the same removal process as any other Member.

	\subsection{Section 4 - Replacing Officers}

	If an Officer is either removed or resigns during their term, the remaining Officers shall replace the missing Officer according in a manner similar to the initial Officer electing process.

	\subsection{Section 5 - Advisors}

	The Advisor(s) should be a person of technical experience. Preferably someone who has been involved in the open source ecosystem. The Advisor(s) must grok the goals of The Open Source Club so that they do not get in the way of cool things. The Advisor(s) exist to provide guidance, mentorship, and cluefulness. Any Advisor's term is one academic year.

	University guidelines demand that the Advisor(s) (sub-advisors, and/or co-advisors if applicable) ``of student organizations must be full-time members of The Ohio State University faculty, administrative, professional, or general staff.'' In addition to that, The Open Source Club also requires that any student who is enrolled in any classes at The Ohio State University not be eligible to become or currently be an Advisor.

	\section{Article IV - Systems}

  The Open Source Club shall retain computer systems insofar as they are used to further the stated purpose of The Open Source Club. All computer systems that the Open Source Club administers \textit{must} be open source.

  \subsection{Section 1 - Hosting}

  When hosting code, individual software projects must be allowed to be managed however the individuals involved in the project see fit. The Open Source Club shall not create strict guidelines for:

  \begin{itemize}
    \item Any ``official license'',
    \item What kind of software should be developed,
    \item How a software project should be managed (including frequency of updates, release models, code repository management, etc).
  \end{itemize}

  However, The Open Source Club shall not host any code or content that would be categorized as illegal or against The Ohio State University's policies.

  \subsection{Section 2 - Access Management}

	Accounts on The Open Source Club computers can be given out to any Member who requests an account, provided Ohio State University's rules are followed.  These accounts may be revoked due to inactivity or failure to follow the rules. Accounts must be revoked if a Member loses their membership status.

  Key card access to The Open Source Club office can be granted on request given the consent of the president, provided that Ohio State University rules are followed.  Access may be revoked due to inactivity or failure to follow the rules. Access must be revoked if a Member loses their membership status.

  \subsection{Section 3 - Operations}

  The computer systems of The Open Source Club must be maintained with a reasonable degree of care. Any service that is offered to general Members of The Open Source Club which is hosted by The Open Source Club must be reasonably:  

  \begin{itemize}
    \item Backed up in an efficient and effective way,
    \item Secured in such a way that protects the \textit{totality} of the Members' and systems' data,
    \item Kept up-to-date with all patches, security or otherwise.
    \item Kept in compliance with The Ohio State University's registration and other requirements
  \end{itemize}

	\section{Article V - Meetings}

  The Open Source Club is to have meetings once a week through the entirety of all Fall and Spring semesters on all weeks wherein the given day of the week to meet does not fall on a holiday or break which is observered by The Ohio State University. If none of the Officers can be in attendance at a given meeting then said meeting will not formally take place, as decisions cannot be made without the presiding Officers. Members can still hold an independent meeting, but it will not be recognized as an official meeting.

  \subsection{Section 1 - Types}

  All of the meetings must revolve around one or more aspects of the open source ecosystem. They also must be in the form of a panel, tutorial, or scripted presentation. Meetings are not a time for unstructured forums, study sessions, or project development. Despite that, participation by and questions from Members is \textit{highly} encouraged.

  \subsection{Section 2 - Presentors}

  Examples of acceptable presentors are:

  \begin{itemize}
    \item Members
    \item Alumni of The Open Source Club
    \item Representatives of Companies
    \item Professors
    \item Advisor(s)
  \end{itemize}
  
  \subsection{Section 2 - Rationale}

  The rationale for The Open Source Club to hold meetings is twofold. Primarily, meetings are the predominant vehicle in which to fulfill the Purpose of this Document. Additionally, meetings serve to connect the Members into a network of connections throughout the local and global community. All meetings must strive to simultaneously maximize the emphasis on both of these reasons. These are why meetings are held in the first place.

	\section{Article VI - This Document}

  \subsection{Section 1 - Validity}

  This Document shall be valid insofar as it is itself open source.

  \subsection{Section 2 - Rationale for Changes}

	In the case that someone thinks that they need to alter the constitution for bug fixes, adding features, or compatibility with new hardware, the election must be announced at least a week in advance and the vote must be carried by a supermajority. Amendments and changes should be taken advisedly and considered for a reasonable amount of time before being implemented.

	It is strictly encouraged that the constitution should be changed frequently. If it is to be amended, then this article (article VI) allows the creation or modification of any by-laws in favor of amending the main body of the Constitution (this Document) with a majority vote.

	It is further strictly \textit{discouraged} that any amendments or by-laws restricting free commerce and creativity be created. New amendments or by-laws should not conflict with the above stated purpose of The Open Source Club.

  \subsection{Section 3 - Versioning}

  This Document should be versioned using semantic versioning for in so long as semantic versioning shall be considered open source. Semantic versioning for this Document is defined as:

  \begin{itemize}
    \item A MINOR Versioning \textit{must} be done whenever a voting Member supermajority occurs that makes incompatible API changes regarding the formatting of this Document,
    \item A MINOR Versioning when a voting Member majority adds content in a backwards-compatible manner,
    \item A PATCH Versioning when a voting Member majority makes backwards-compatible bug fixes.
  \end{itemize}

	\section{Article VII - Method of Dissolution of Organization}

	Should The Open Source Club be forced to dissolve itself, any and all assets should be put toward The Open Source Club's debt (if any). This includes equipment and/or funds owed to or owned by The Ohio State University.

	The remaining assets (hardware, operating funds, etc) should be donated to an open source organization, like the FSF or the OSI. The organization to which the assets are donated must be determined at the time of dissolution by the Benevolent Dictator, with final approval of the Advisor(s).

	\begin{center}
		\includegraphics [height=2.5em] {cc-0.png}
	\end{center}

\end{document}
